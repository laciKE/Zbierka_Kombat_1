\nazov{Binomické koeficienty}

\pr Vypočítajte: $\sum_{k\geq0} {n \choose k}\frac{k!}{\left(n+1+k\right)!}$, $n\in \mathbb{N}_0$.
\ries{Stanek}
\begin{eqnarray*}
\sum_{k\geq0} {n \choose k}\frac{k!}{\left(n+1+k\right)!} &=& \sum_{k\geq0} \frac{n!}{k!\left(n-k\right)!}\frac{k!}{\left(n+1+k\right)!} = \sum_{k\geq0}\frac{n!}{\left(n-k\right)!\left(n+1+k\right)!} =\\
&=& \sum_{k\geq0}\frac{n!}{\left(n-k\right)!\left(n+1+k\right)!}\frac{\left(2n+1\right)!}{\left(2n+1\right)!} = \frac{n!}{\left(2n+1\right)!}\sum_{k\geq0} {2n+1 \choose n-k} =\\
&=& \frac{n!}{\left(2n+1\right)!}\sum_{k=0}^n {2n+1 \choose k} = \frac{n!}{\left(2n+1\right)!}\left(\frac12 2^{2n+1}\right) = \frac{n!}{\left(2n+1\right)!}2^{2n}
\end{eqnarray*}

\pr Vypočítajte: $\sum_{k \leq n}{n \choose k}2^{k-n}$, $n \in \mathbb{Z}$.
\ries{Laco} Ak $n<0$, potom aj $k<0$ a všetky kombinačné čísla sú nulové, takže aj výsledná suma je nulová. Ak $n=0$, tak jedine pre $k=0$ obsahuje suma nenulový člen a je rovná $1$. Ak $n>0$, môžeme použiť binomickú vetu:
$$
\sum_{k \leq n}{n \choose k}2^{k-n} = \frac1{2^n} \sum_{k=0}^n{n \choose k}2^k = \frac1{2^n} \left(1+2\right)^n = \left(\frac32\right)^n.
$$

\pr Vypočítajte: $\sum_{k \geq n}{k \choose n}2^{n-k}$, $n \in \mathbb{Z}$.
\ries{Stanek} Ak $n<0$, dolný index v každom kombinačnom čísle je záporný a teda hodnota celej sumy je nula. Ďalej budeme predpokladať, že $n\geq0$.
\begin{eqnarray*}
\sum_{k \geq n}{k \choose n}2^{n-k} &=& \sum_{n\leq k}{k-n-k-1 \choose k-n}\left(-1\right)^{k-n}2^{n-k} = \sum_{n\leq k}{-n-1 \choose k-n}\left(-2\right)^{n-k} = \\
&=& \left/\begin{array}{l}
l = k-n\\
k = n+l
\end{array}\right/ = \sum_{n\leq n+l}{-n-1 \choose l+n-n}\left(-2\right)^{n-l-n} = \sum_{0 \leq l} {-n-1\choose l}\left(-2\right)^{-l} = \\
&=& \sum_{0\leq l}{-n-1 \choose l} \left(-\frac12\right)^l = \left(1 - \frac12\right)^{-n-1} = 2^{n+1}
\end{eqnarray*}

\pr Vypočítajte: $\sum_k{-\frac12 \choose k}\left(\frac12\right)^k k$.
\ries{Laco}
\begin{eqnarray*}
\sum_k{-\frac12 \choose k}\left(\frac12\right)^k k &=& \sum_k\frac{-\frac12}k{-\frac32 \choose k-1}\left(\frac12\right)^k k = -\frac14 \sum_k{-\frac32 \choose k-1}\left(\frac12\right)^{k-1} = \\
&=& -\frac14 \sum_{k-1}{-\frac32 \choose k-1}\left(\frac12\right)^{k-1} = -\frac14\left(1+\frac12\right)^{-\frac32} = -\frac16 \sqrt{\frac23}
\end{eqnarray*}

\pr Vypočítajte: $\sum_{l\geq 2}\left(-1\right)^l{\dolna{e^m} \choose m-l}$, $m\in\mathbb{Z}$.
\ries{Laco} Aby v sume bol aspoň jeden dolný index v ${\dolna{e^m} \choose m-l}$ nezáporný, tak $m\geq2$, inak je hodnota sumy $0$. Pre dané $m\geq2$ označme $\dolna{e^m} = c$, pričom vieme, že $c\geq\dolna{e^2}\geq\dolna{2^2}\geq4$. Zároveň v sume nám stačí uvažovať iba $m-l\geq 0$, teda $2\leq l \leq m$.
\begin{eqnarray*}
\sum_{l=2}^m\left(-1\right)^l{c \choose m-l} &=& \left/ k=m - l\right/ = \sum_{k=0}^{m-2}\left(-1\right)^{m-k}{c \choose k} = \left(-1\right)^m\sum_{k=0}^{m-2}\left(-1\right)^k\left({c-1 \choose k-1} + {c-1 \choose k}\right) =\\
&=& \left(-1\right)^m\left\lbrack{c-1 \choose -1} + {c-1 \choose 0} - {c-1 \choose 0} - \cdots + \left(-1\right)^{m-2}\left({c-1 \choose m-3} + {c-1 \choose m-2}\right) \right\rbrack = \\
&=& \left(-1\right)^{2m-2}{c-1 \choose m-2} = {\dolna{e^m}-1 \choose m-2}
\end{eqnarray*}

\pr Vypočítajte: $\sum_{k=0}^{2n}{n \choose k}\frac1{k+2}$, $n\in\mathbb{N}_0$.
\ries{Stanek}
\begin{eqnarray*}
\sum_{k=0}^{2n}{n \choose k}\frac1{k+2} &=& \sum_{k=0}^{2n}{n \choose k}\frac1{k+2} \frac{\left(k+1\right)\left(n+1\right)\left(n+2\right)}{\left(k+1\right)\left(n+1\right)\left(n+2\right)} = \frac1{\left(n+1\right)\left(n+2\right)}\sum_{k=0}^{2n}{n+2 \choose k+2}\left(k+1\right) = \\
&=& \frac1{\left(n+1\right)\left(n+2\right)}\left(\sum_{k=0}^{2n}{n+2 \choose k+2}\left(k+2\right) - \sum_{k=0}^{2n}{n+2 \choose k+2}\right) = \\
&=& \frac1{\left(n+1\right)\left(n+2\right)}\left(\left(n+2\right)\sum_{k=0}^{2n}{n+1 \choose k+1}- \sum_{k=0}^{2n}{n+2 \choose k+2}\right) = \\
&=& \frac1{\left(n+1\right)\left(n+2\right)}\left(\left(n+2\right)\sum_{k=1}^{2n+1}{n+1 \choose k} - \sum_{k=2}^{2n+2}{n+2 \choose k}\right) =\\
&=& \frac1{\left(n+1\right)\left(n+2\right)}\left(\left(n+2\right)\left(2^{n+1}-{n+1 \choose 0}\right) - \left(2^{n+2} - {n+2 \choose 0} - {n+2 \choose 1}\right)\right) =\\
&=& \frac{2^{n+1}-1}{n+1} - \frac{2^{n+2} - n - 3}{\left(n+1\right)\left(n+2\right)}
\end{eqnarray*}

\pr Vypočítajte: $\sum_{k=0}^{n}{n \choose k}\left(-1\right)^k k^2$, $n\in\mathbb{N}_0$.
\ries{Stanek}
\begin{eqnarray*}
\sum_{k=0}^{n}{n \choose k}\left(-1\right)^k k^2 &=& n\sum_{k=0}^n {n-1 \choose k-1} k \left(-1\right)^k = n\left(\sum_{k=0}^n {n-1 \choose k-1} \left(k-1\right) \left(-1\right)^k + \sum_{k=0}^n {n-1 \choose k-1} \left(-1\right)^k \right) = \\
&=& \left/ \sum_{k=0}^n {n-1 \choose k-1} \left(-1\right)^k = 0 \hbox{ pre } n\geq2\right/ = n\left(n-1\right)\sum_{k=0}^n{n-2 \choose k-2}\left(-1\right)^k = 0 \hbox{ pre } n\geq3
\end{eqnarray*}
Pre $n\geq3$ je suma rovná $0$, pre $n\in\left\{0,1,2\right\}$ jej hodnotu spočítame dosadením.

\pr Vypočítajte: $\sum_{k=0}^{n}{n+1 \choose k}\left(\frac{n+1}m\right)^k \left(-1\right)^{n+k}$, $n,m\in\mathbb{N}_0$.
\ries{Laco} Najprv upravím podmienku, $m$ nemôže byť $0$, inak by zlomok v sume nebol definovaný, takže ďalej už uvažujem len $m>0$.
\begin{eqnarray*}
\sum_{k=0}^{n}{n+1 \choose k}\left(\frac{n+1}m\right)^k \left(-1\right)^{n+k} &=& \left/ x = \frac{n+1}m \right/ = \left(-1\right)^n\sum_{k=0}^n{n+1 \choose k}x^k\left(-1\right)^k =\\
&=& \left(-1\right)^n\left(\sum_{k=0}^{n+1}{n+1 \choose k}x^k\left(-1\right)^k - {n+1 \choose n+1}x^{n+1}\left(-1\right)^{n+1}\right) = \\
&=& \left(-1\right)^n\left(1-x\right)^{n+1} - \left(-1\right)^{2n+1}x^{n+1} = \left(-1\right)^n\left(1-x\right)^{n+1}+x^{n+1} = \\
&=& \left(-1\right)^n\left(\frac{m-n-1}m\right)^{n+1}+\left(\frac{n+1}m\right)^{n+1}
\end{eqnarray*}

\pr Vypočítajte: $\sum_{k\geq1}\frac{n+1}k{n+1 \choose 2k-1}$, $n\in\mathbb{N}$.
%\ries{} TODO

\pr Vypočítajte: $\sum_k{n \choose 2n-k}{-n+k \choose k+2n+p}{k-3p \choose 2m+k}\left(-1\right)^k$, $m,n,p\in\mathbb{Z}$, $m\geq n\geq 0$.
\ries{Stanek}
\begin{eqnarray*}
\sum_k{n \choose 2n-k}{-n+k \choose k+2n+p}{k-3p \choose 2m+k}\left(-1\right)^k &=& \left/ {n \choose p}{p \choose 2n-k} = {n \choose 2n-k}{-n+k \choose k-2n+p}\right/ = \\
&=& {n \choose p}\sum_k{p \choose 2n-k}{2m+k-k+3p-1 \choose 2m+k} = \left/\hbox{konvolúcia}\right/\\
&=& {n \choose p}{2m+4p-1 \choose 2n+2m}
\end{eqnarray*}

